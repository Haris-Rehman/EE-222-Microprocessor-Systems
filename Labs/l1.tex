\documentclass{article}

\usepackage{graphicx}

\title{Lab 01: Programming in Assembly Language using MASM \\ \large EE-222: Microprocessor System}
\begin{document}
\maketitle
\tableofcontents
\newpage
\begin{abstract}
Microprocessor architecture is the actually the interface in between software and hardware. A microprocessor system consists of \emph{Microarchitecture and ISA}, latter begin software while prior the hardware layer. \\
Since ISA is a part of microarcitectury, it is surely hardware-dependent as well. Each microarchitecture has its own ISA defined with its own pros and cons, as per trade-offs decided by the architect. \\
Here we explore the all-famous \emph{x86 ISA} which is the intruction set of \emph{80x86 series}, the core of this course.
\newpage
\end{abstract}
\section{MASM}
The Microsoft Macro Assembler (MASM) is an x86 assembler for Microsoft Windows that uses the Intel syntax. Assembly language is a great tool to understand how a computer works and with the help of MASM you will be able to assemble and run your programs written in Assembly language.
\section{Assembly Language}
\subsection{Addition Code}
\begin{verbatim}
TITLE Add two registers (example.asm)
; The comments are given after the semi colon on a line
; This program adds 32-bit unsigned
; integers and stores the sum in the ecx register
Include irvine32.inc
.data
;variable declarations go here
.code
Main Proc
;instructions go here
Mov 	eax, 30		;Assembly Language is NOT case sensitive
Mov 	ebx, 20
Add 	edx, eax
Add 	edx, ebx
Call 	dumpregs	;displays the result on the screen by displaying all register values
Exit
Main endp
End main
\end{verbatim}
\subsection{Execution}
\includegraphics[scale=1]{out}
\newpage
\section{Debugging}
\subsection{Analysis}
\begin{center}
\begin{tabular}{c c c}
	Resister & Content & Decimal\\
	EAX & 0000001E & 30 \\
	EBX & 00000014 & 20 \\
	EDX & 00401037 & 4198455
\end{tabular}
\end{center}
\subsection{Verification}
\begin{itemize}
\item MOV
	\begin{itemize}
	\item EAX stores 30
	\item EBX stored 20
	\end{itemize}
\item ADD
	\begin{itemize}
	\item The add command stores a garbage value in registem EDX
	\end{itemize}
\end{itemize}
\paragraph{Reason}
The possible reason for this error is \emph{already existing garbage value} in registex EDX
\paragraph{Solution}
This poblem can be resolved by \emph{storing zero} in register EDX before addition. Thus garbage value will be cleared.
\subsection{Modified Code}
\begin{verbatim}
TITLE Add two registers (example.asm)
; The comments are given after the semi colon on a line
; This program adds 32-bit unsigned
; integers and stores the sum in the ecx register
Include irvine32.inc
.data
;variable declarations go here
.code
Main Proc
;instructions go here
Mov		edx, 0		;erasing the garbage value
Mov 	eax, 30	
Mov 	ebx, 20
Add 	edx, eax
Add 	edx, ebx
Call 	dumpregs	
Exit
Main endp
End main
\end{verbatim}
\subsection{Debugged Execution}
\includegraphics[scale=1]{out2}
\subsection{Analysis}
\begin{center}
\begin{tabular}{c c c}
	Resister & Content & Decimal\\
	EAX & 0000001E & 30 \\
	EBX & 00000014 & 20 \\
	EDX & 00000032 & 50
\end{tabular}
\end{center}
\section{Summary}
The basic code in x86 version of assembly language has been debugged using hit-and-trial, and deductive method using DOS command-line assembler MASM.
\end{document}